\documentclass[a4paper,twoside]{article}
\usepackage{graphicx}
\usepackage{lineno}
\usepackage{url}
\usepackage[natbib=true]{biblatex}
\addbibresource{o2bibliography.bib}

\linenumbers

\def\O2{O$^2$}

\begin{document}

\title{ALICE \O2 data model proposal}

\author{Mikolaj Krzewicki \and Sylvain Chapelain \and Roberto Divia \and Matthias Richter \and David Rohr}
\date{
for the CWG4 data model group.\\[2ex]
\today
}
\maketitle

\section{Introduction}

The ALICE online-offline (\O2) computing system \cite{o2tdr,o2} is a computing facility and a software framework designed for the processing of the ALICE data in the upcoming LHC Run 3.
The design aims at high data throughput and parallelism using a multiprocess model.
It does not, however, exclude the use of multithreading and other forms of concurrent processing inside of individual processes.

The data exchange between processes running within the \O2 system (called \O2 devices) is taken care of by the ALICE-FAIR (Alfa ) framework \cite{alfa}. Since this is the only communication mechanism foreseen for data exchange, it effectively serves the role of an API between the devices.
The Alfa framework provides data transport and synchronisation primitives via the FairMQ message queue library. FairMQ messages consist of raw memory buffers which are asynchronously queued and atomically delivered.

The data processed by the \O2 system consists of a set of buffers originating from both the detector hardware and the processing components. The data fragments are logically grouped into (sub-) time frames. A (sub-) time frame contains the data associated to a period of data taking (typically several tens of ms) and/or the results of processing of these data plus any additional metadata that might be necessary to describe and qualify the data set.

\section{Vectored IO}

Vectored IO is an important feature when dealing with multiple data buffers as it allows, in principle, to avoid the cost associated with serializing data into a single IO buffer.
Vectored IO in FairMQ is provided in the form of multi-part messages consisting of multiple buffers which are delivered atomically while preserving the initial ordering.


\section{Message structure}

\section{Metadata format}

\section{Data formats}

\section{Interfaces}

\printbibliography

\end{document}
